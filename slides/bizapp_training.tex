<<<<<<< HEAD
% Created 2014-09-26 金 13:12
=======
% Created 2014-09-26 金 11:48
>>>>>>> d35d84d85541f89a6cd5bf5abc516343597916a0
\documentclass[t, aspectratio=169]{beamer}
\usepackage{zxjatype}
\usepackage[ipa]{zxjafont}
\setbeamertemplate{navigation symbols}{}
\hypersetup{colorlinks,linkcolor=,urlcolor=gray}
\AtBeginPart
{
  \begin{frame}<beamer|handout>
    \date{\insertpart}
    \maketitle
  \end{frame}
}
\AtBeginSection[]
{
  \begin{frame}<beamer>
  \tableofcontents[currentsection,currentsubsection]
  \end{frame}
}

\usepackage{minted}
\institute[AIIT]{産業技術大学院大学(AIIT)}
\usetheme{Berkeley}
\usecolortheme{seahorse}
\useinnertheme{rectangles}
\author{中鉢 欣秀・上田 隆一}
\date{2014-09-22}
\title{ビジネスアプリケーション演習}
\hypersetup{
  pdfkeywords={},
  pdfsubject={},
  pdfcreator={Emacs 24.3.2 (Org mode 8.2.5h)}}
\begin{document}

\maketitle


\part{第1章 モダンなソフトウエア開発の道具達}
\label{sec-1}
\section{連絡事項}
\label{sec-1-1}
\begin{frame}[fragile,label=sec-1-1-1]{連絡事項}
 \begin{block}{資料等の入手先}
\begin{itemize}
\item GitHubの下記リポジトリにまとめておきます
\begin{itemize}
\item \href{https://github.com/ychubachi/enpit}{ychubachi/enpit}
\end{itemize}
\item 資料は随時updateするので,適宜,最新版をダウンロードしてください
\end{itemize}
\end{block}

\begin{block}{Twitterのハッシュタグ}
\begin{itemize}
\item Twitterのハッシュタグは \texttt{\#enpit\_aiit} を使ってください
\item まとめサイトなど作ってくれると嬉しいです
\begin{itemize}
\item 昨年の例 -> \href{http://togetter.com/li/558221}{enPiT BizApp AIIT ビジネスアプリケーション演習 1日目 - Togetterまとめ}
\end{itemize}
\end{itemize}
\end{block}
\end{frame}

\section{授業の全体像}
\label{sec-1-2}
\begin{frame}[label=sec-1-2-1]{学習目標と目的}
\begin{block}{目標}
\begin{itemize}
\item ビジネスアプリケーションを構築するための基礎力
\item 分散型PBLを実施する上で必要となる知識やツールの使い方
\item これら活用するための自己組織的なチームワーク
\end{itemize}
\end{block}

\begin{block}{目的}
\begin{itemize}
\item 分散ソフトウェア開発のための道具を学ぶ
\begin{itemize}
\item 開発環境(Ruby),VCSとリモートリポジトリ(GitHub)
\item テスト自動化,継続的インテグレーション,PaaS
\end{itemize}
\end{itemize}
\end{block}
\end{frame}

\begin{frame}[label=sec-1-2-2]{前提知識と到達目標}
\begin{block}{前提とする知識}
\begin{itemize}
\item 情報系の学部レベルで基礎的な知識を持っていること
\end{itemize}
\end{block}

\begin{block}{最低到達目標}
\begin{itemize}
\item 授業で取り上げる各種ツールの基本的な使い方を身につける
\end{itemize}
\end{block}

\begin{block}{上位到達目標}
\begin{itemize}
\item 授業で取り上げる各種ツールの高度な使い方に習熟する.
\end{itemize}
\end{block}
\end{frame}

\begin{frame}[label=sec-1-2-3]{授業の形態}
\begin{block}{対面授業}
\begin{itemize}
\item 担当教員による講義・演習
\end{itemize}
\end{block}

\begin{block}{個人演習}
\begin{itemize}
\item 個人によるソフトウエア開発
\end{itemize}
\end{block}

\begin{block}{グループ演習}
\begin{itemize}
\item グループによるソフトウエア開発
\end{itemize}
\end{block}
\end{frame}

\section{授業の方法}
\label{sec-1-3}
\begin{frame}[label=sec-1-3-1]{講義・演習・課題}
\begin{block}{講義}
\begin{itemize}
\item ツールの説明
\item ツールの使い方
\end{itemize}
\end{block}

\begin{block}{演習}
\begin{itemize}
\item 個人でツールを使えるようになる
\item グループでツールを使えるようになる
\end{itemize}
\end{block}
\end{frame}

\begin{frame}[label=sec-1-3-2]{成績評価}
\begin{block}{課題}
\begin{itemize}
\item 個人でソフトウエアを作る
\item グループでソフトウエアを作る
\end{itemize}
\end{block}

\begin{block}{評価の方法}
\begin{itemize}
\item 課題提出と実技試験
\end{itemize}
\end{block}

\begin{block}{評価の観点}
\begin{itemize}
\item 分散PBLで役に立つ知識が習得できたかどうか
\end{itemize}
\end{block}
\end{frame}

\section{モダンなソフトウエア開発とは}
\label{sec-1-4}
\begin{frame}[label=sec-1-4-1]{ソフトウエア開発のための方法・言語・道具}
\begin{figure}[htb]
\centering
\includegraphics[width=0.6\textwidth]{./figures/FLT_framework.pdf}
\caption{\label{FLT_framework}The Framework-Language-Tool framework.}
\end{figure}
\end{frame}

\begin{frame}[label=sec-1-4-2]{授業で取り上げる範囲}
\begin{block}{取り上げること}
\begin{itemize}
\item 方法を支えるための道具
\item 良い道具には設計概念として方法論が組み込まれている
\item 道具はプログラミング言語を問わない
\end{itemize}
\end{block}

\begin{block}{取り扱わないこと}
\begin{itemize}
\item 方法論そのものについてはアジャイル開発特論で学ぶ
\item 言語の備えるエコシステムについては必要な範囲で学ぶ
\end{itemize}
\end{block}
\end{frame}

\begin{frame}[label=sec-1-4-3]{Scrumするための道具}
\begin{figure}[htb]
\centering
\includegraphics[width=0.6\textwidth]{./figures/tools.pdf}
\caption{\label{tools}The modern tools for Scrum developments.}
\end{figure}
\end{frame}
\begin{frame}[label=sec-1-4-4]{モダンな開発環境の全体像}
\begin{block}{仮想化技術(Virtualization)}
\begin{itemize}
\item WindowsやMacでLinux上でのWebアプリケーション開発を学ぶことができる
\item HerokuやTravis CI等のクラウドでの実行や検査環境として用いられている
\end{itemize}
\end{block}

\begin{block}{ソーシャルコーディング(Social Coding)}
\begin{itemize}
\item LinuxのソースコードのVCSとして用いられているGitを学ぶ
\item GitはGitHubと連携することでOSS型のチーム開発ができる
\end{itemize}
\end{block}
\end{frame}
\begin{frame}[label=sec-1-4-5]{enPiT仮想化環境}
\begin{block}{インストール済みの言語と道具}
\begin{itemize}
\item エディタ(Emacs/Vim)
\item Rubyの実行環境
\item GitHub,Heroku,Travis CIと連携するための各種コマンド(github-connect.sh,hub,heroku,travis)
\item PostgreSQLのクライアント・サーバーとDB
\item 各種設定ファイル(.bash\_profile,.gemrc,.gitconfig)
\item その他
\end{itemize}
\end{block}

\begin{block}{仮想化環境の構築用リポジトリ(参考)}
\begin{itemize}
\item \href{https://github.com/ychubachi/vagrant_enpit}{ychubachi/vagrant\_enpit}
\end{itemize}
\end{block}
\end{frame}

\section{<演習課題1> (準備作業)}
\label{sec-1-5}
\begin{frame}[label=sec-1-5-1]{クラウドのアカウント作成}
\begin{block}{GitHub}
\begin{itemize}
\item\relax [\href{https://github.com/join}{Join GitHub · GitHub}]
\end{itemize}
\end{block}

\begin{block}{Heroku}
\begin{itemize}
\item\relax [\href{https://id.heroku.com/signup}{Heroku - Sign up}]
\end{itemize}
\end{block}

\begin{block}{Travis CI}
\begin{itemize}
\item\relax [\href{https://travis-ci.org/}{Travis CI}]
\begin{itemize}
\item Travis CIは,GitHubのアカウントでログインできる
\end{itemize}
\end{itemize}
\end{block}
\end{frame}

\begin{frame}[fragile,label=sec-1-5-2]{enPiT仮想化環境のアップデート}
 \begin{block}{作業内容}
\begin{itemize}
\item enPiT仮想化環境(vagrantのbox)を更新しておく
\end{itemize}
\end{block}

\begin{block}{コマンド}
\begin{minted}[]{bash}
cd ~/enpit
vagrant destroy
vagrant box update
\end{minted}
\end{block}
\end{frame}

\begin{frame}[fragile,label=sec-1-5-3]{Port Forwardの設定}
 \begin{block}{説明}
\begin{itemize}
\item Guest OSで実行するサーバに,Host OSからWebブラウザでアクセスできるようにしておく
\item 任意のエディタでVagrantfileを変更
\end{itemize}
\end{block}

\begin{block}{変更前}
\begin{minted}[]{ruby}
# config.vm.network "forwarded_port", guest: 80, host: 8080
\end{minted}
\end{block}
\begin{block}{変更後}
\begin{minted}[]{ruby}
config.vm.network "forwarded_port", guest: 3000, host: 3000
config.vm.network "forwarded_port", guest: 4567, host: 4567
\end{minted}
\end{block}
\end{frame}
\begin{frame}[fragile,label=sec-1-5-4]{enPiT仮想化環境にログイン}
 \begin{block}{作業内容}
\begin{itemize}
\item 前の操作に引き続き,仮想化環境にSSH接続する
\end{itemize}
\end{block}

\begin{block}{コマンド}
\begin{minted}[]{bash}
vagrant up
vagrant ssh
\end{minted}
\end{block}
\end{frame}
\begin{frame}[label=sec-1-5-5]{github-connectスクリプト}
\begin{block}{URL}
\begin{itemize}
\item\relax [\href{https://gist.github.com/ychubachi/6491682}{github-connect.sh}]
\end{itemize}
\end{block}

\begin{block}{git conifgを代行}
\begin{itemize}
\item GitHubにログインし,名前とemailを読み込んでgitに設定
\end{itemize}
\end{block}

\begin{block}{SSHの鍵生成と登録}
\begin{itemize}
\item SSH鍵を作成し,公開鍵をGitHubに登録してくれる
\end{itemize}
\end{block}
\end{frame}

\begin{frame}[fragile,label=sec-1-5-6]{github-connect.shの実行}
 \begin{block}{作業内容}
\begin{itemize}
\item スクリプトを起動し,設定を行う
\item GitHubのログイン名とパスワードを聞かれるので,入力する
\item rsa key pairのパスフレーズは入力しなくて構わない
\end{itemize}
\end{block}

\begin{block}{コマンド}
\begin{minted}[]{bash}
github-connect.sh
\end{minted}
\end{block}
\end{frame}
\begin{frame}[fragile,label=sec-1-5-7]{GitとGitHubの設定確認}
 \begin{block}{Gitの設定確認}
\begin{minted}[]{bash}
git config --list
\end{minted}
\end{block}

\begin{block}{GitHubの設定確認}
\begin{itemize}
\item ブラウザでGitHubのSSH Keyページを開く
\end{itemize}
\end{block}
\end{frame}
\part{第2章 Git/GitHubの基本操作}
\label{sec-2}
\section{ローカルリポジトリ}
\label{sec-2-1}
\begin{frame}[fragile,label=sec-2-1-1]{Gitのローカルリポジトリの作成}
 \begin{block}{ローカルリポジトリ}
\begin{itemize}
\item ソースコードや各種のファイルを保存し,開発に利用する
\item 「 \texttt{my\_enpit} 」というディレクトリを作成し,初期化する
\end{itemize}
\end{block}

\begin{block}{コマンド}
\begin{minted}[]{bash}
mkdir ~/my_enpit
cd ~/my_enpit
git init
\end{minted}
\end{block}
\end{frame}
\begin{frame}[fragile,label=sec-2-1-2]{Gitの設定ディレクトリ}
 \begin{block}{隠しフォルダ「 \texttt{.git} 」}
\begin{itemize}
\item Gitソースコードの履歴情報や,各種の設定をGitが保存するディレクトリ
\item このフォルダは通常,Gitを経由しないで変更することはない
\end{itemize}
\end{block}

\begin{block}{確認方法}
\begin{minted}[]{bash}
ls -a
find .git
\end{minted}
\end{block}
\end{frame}
\section{リモートリポジトリ}
\label{sec-2-2}
\begin{frame}[fragile,label=sec-2-2-1]{Hubコマンド}
 \begin{block}{enPiT環境のHubコマンド}
\begin{itemize}
\item \href{https://github.com/github/hub}{github/hub}
\end{itemize}
\end{block}

\begin{block}{GitへのGitHub操作機能追加}
\begin{itemize}
\item 通常のGitの機能に加えて,GitHub用のコマンドが利用できる
\item エイリアス設定しており,コマンド名は「git」のまま
\end{itemize}
\end{block}

\begin{block}{確認方法}
\begin{minted}[]{bash}
git version
alias git
\end{minted}
\end{block}
\end{frame}
\begin{frame}[fragile,label=sec-2-2-2]{Hubコマンドによるリモートリポジトリの作成}
 \begin{block}{作業内容}
\begin{itemize}
\item コマンドライン操作で,GitHubにリポジトリを作成する
\item Hubコマンドの機能である \texttt{git create} を利用
\item 初回既動時にはパスワードか聞かれる
\end{itemize}
\end{block}

\begin{block}{コマンド}
\begin{minted}[]{bash}
git create
\end{minted}
\end{block}
\end{frame}
\begin{frame}[fragile,label=sec-2-2-3]{リポジトリの確認方法}
 \begin{block}{確認方法}
\begin{itemize}
\item WebブラウザでGitHubを開き,「 \texttt{my\_enpit} 」ができていることを確認
\end{itemize}
\end{block}

\begin{block}{コマンドラインで確認}
\begin{minted}[]{bash}
git remote -vv
\end{minted}
\end{block}
\end{frame}

\section{GitとGitHubの基本操作}
\label{sec-2-3}
\begin{frame}[label=sec-2-3-1]{Gitの操作方法}
\begin{block}{マニュアル等}
\begin{itemize}
\item \href{http://git-scm.com/doc}{Git - Documentation}
\end{itemize}
\end{block}

\begin{block}{commitログの書き方}
\begin{itemize}
\item \href{https://github.com/erlang/otp/wiki/Writing-good-commit-messages}{Writing good commit messages · erlang/otp Wiki}
\end{itemize}
\end{block}
\end{frame}

\begin{frame}[fragile,label=sec-2-3-2]{ステータスの確認}
 \begin{block}{リポジトリの状態を確認する}
\begin{itemize}
\item \texttt{git status} は,頻繁に利用するコマンド
\item リポジトリの状態を確認することができる
\item この表示の読み方を理解することが重要
\end{itemize}
\end{block}

\begin{block}{コマンド}
\begin{minted}[]{bash}
git status
\end{minted}
\end{block}
\end{frame}
\begin{frame}[fragile,label=sec-2-3-3]{ファイルの追加とステータスの確認}
 \begin{block}{作業内容}
\begin{itemize}
\item テキストエディタで \texttt{README.md} を作成
\item ステータスの変化を見る
\end{itemize}
\end{block}

\begin{block}{コマンド}
\begin{minted}[]{bash}
emacs README.md
git status
\end{minted}
\end{block}
\end{frame}
\begin{frame}[label=sec-2-3-4]{Add/Commitの方法}
\begin{block}{ステージングエリアを利用する場合}
\begin{itemize}
\item git add README.mb
\item git commit -m 'First commit'
\end{itemize}
\end{block}

\begin{block}{ステージングエリアを省略する場合}
\begin{itemize}
\item git commit -a -m 'First commit'
\begin{itemize}
\item トラックされていないファイルはcommitしないので注意
\end{itemize}
\end{itemize}
\end{block}
\end{frame}
\begin{frame}[fragile,label=sec-2-3-5]{リモートリポジトリへの公開}
 \begin{block}{pushとは?}
\begin{itemize}
\item ローカルで作成したcommitを,リモートのリポジトリにアップロードすること
\item originとは,リモートのリポジトリの内部的な名前
\item upstreamとは,ブランチ(後述)が紐づいているリポジトリのこと
\item 最初にそのブランチをpushするときは, \texttt{-{}-setupstream} オプションを指定
\end{itemize}
\end{block}

\begin{block}{コマンド}
\begin{minted}[]{bash}
git push --set-upstream origin master
\end{minted}
\end{block}
\end{frame}
\begin{frame}[fragile,label=sec-2-3-6]{Logの閲覧}
 \begin{block}{コミットログ}
\begin{itemize}
\item ソースコードに加えた変更の履歴を,commitを単位として閲覧できる
\end{itemize}
\end{block}

\begin{block}{コマンド}
\begin{minted}[]{bash}
git log
\end{minted}
\end{block}
\end{frame}
\begin{frame}[fragile,label=sec-2-3-7]{コミットのログを詳細に書く方法}
 \begin{block}{エディタを使ったログの記述}
\begin{itemize}
\item コミットのログや,Pull Requestの記述を,より詳しく書くことができる
\item \texttt{commit} や \texttt{pull\_request} から  \texttt{-m} オプションを外すと,エディタが立ち上がる
\begin{itemize}
\item エディタはemacsを起動するようになっている
\item \texttt{C-x C-s} で保存, \texttt{C-x C-c} で終了
\end{itemize}
\end{itemize}
\end{block}

\begin{block}{コマンド}
\begin{minted}[]{bash}
git commit
git pull_request
\end{minted}
\end{block}
\end{frame}
\section{<演習課題2>}
\label{sec-2-4}
\begin{frame}[label=sec-2-4-1]{Init/Status/Addの練習}
\begin{block}{演習課題}
\begin{enumerate}
\item 解説した手順に従い,my\_enpitリポジトリを作成
\item git statusコマンドを実行
\item README.mdファイルを作成しなさい
\item git statusコマンドを実行し,変化を見なさい
\item commitしなさい.ログを必ず書くこと
\item git statusコマンドを実行し,変化を見なさい
\end{enumerate}
\end{block}
\end{frame}

\begin{frame}[fragile,label=sec-2-4-2]{Commit/Log/Pushの練習}
 \begin{block}{演習課題}
\begin{enumerate}
\item README.mdを修正してcommitしなさい
\item 新しいファイルを作成してcommitしなさい
\item 作業が完了したら,pushしなさい( \texttt{-{}-set-upstream} が必要)
\item コミットがpushされていることをWebブラウザで確認しなさい
\item 作成したファイルを削除してcommitしてpushしなさい
\item エディタを使って,詳細なログを書きなさい
\item その他,自由にcommitの作業を試しなさい
\end{enumerate}
\end{block}
\end{frame}

\begin{frame}[label=sec-2-4-3]{ここまでの課題の提出}
\begin{block}{提出物}
\begin{itemize}
\item 下記のものを提出してください
\begin{itemize}
\item GitHubとHerokuアカウント
\item 作成したmy\_enpitリポジトリのURL
\end{itemize}
\end{itemize}
\end{block}

\begin{block}{提出先}
\begin{itemize}
\item\relax [\href{https://docs.google.com/forms/d/1SiKQqDLQw2YiJieYVS79ywpHIaNC3uI9cNPb_ddhC1Q/viewform?usp=send_form}{enPiT演習アカウント(2014)}]
\end{itemize}
\end{block}
\end{frame}
\part{第3章 GitHubを用いた開発の流れ}
\label{sec-3}
\section{GitHub Flow}
\label{sec-3-1}
\begin{frame}[label=sec-3-1-1]{GitHub Flow (1)}
\begin{enumerate}
\item 思い立ったらブランチ作成
\begin{itemize}
\item 新しい機能追加や,アイディアを試す
\end{itemize}
\item ブランチにコミットを追加
\begin{itemize}
\item 変更点をコミットとして作成
\item コミットのログは,他人が読んでわかるように書く
\end{itemize}
\item Pull Requestを開く
\begin{itemize}
\item コミットについて,意見交換ができる
\item 作業途中でPull Requestを出しても構わない
\end{itemize}
\end{enumerate}
\end{frame}

\begin{frame}[fragile,label=sec-3-1-2]{GitHub Flow (2)}
 \begin{enumerate}
\item 議論とレビュー
\begin{itemize}
\item レビューをしたり,質疑応答をしたりする
\end{itemize}
\item マージしてディプロイ
\begin{itemize}
\item \texttt{master} ブランチにマージする(自動でディプロイ)
\item マージの前にテストしたいときは,ローカルで試す
\end{itemize}
\end{enumerate}
参考文献
\begin{itemize}
\item \href{https://guides.github.com/introduction/flow/index.html}{Understanding the GitHub Flow · GitHub Guides}
\end{itemize}
\end{frame}
\section{ブランチの操作}
\label{sec-3-2}
\begin{frame}[fragile,label=sec-3-2-1]{branchの作成}
 \begin{block}{ブランチとは?}
\begin{itemize}
\item リポジトリにはmasterブランチがある
\item 新しい作業を行う場合,必ずbranchを切る
\end{itemize}
\end{block}

\begin{block}{コマンド}
\begin{minted}[]{bash}
git branch new_branch
git branch -vv
\end{minted}
\end{block}
\end{frame}
\begin{frame}[fragile,label=sec-3-2-2]{branchのcheckout}
 \begin{block}{branchを切り替える}
\begin{itemize}
\item checkoutしてブランチを切り替える
\item ブランチをcommitすることができる
\item 切り替える前に,ブランチでの作業はcommitしておく(stashも可)
\end{itemize}
\end{block}

\begin{block}{コマンド}
\begin{minted}[]{bash}
git checkout new_branch
<編集作業>
git commit -a -m 'Create a new branch'
\end{minted}
\end{block}
\end{frame}
\begin{frame}[fragile,label=sec-3-2-3]{他のbranchをmergeする}
 \begin{block}{mergeとは}
\begin{itemize}
\item ブランチで作業した内容(commit)を,他のブランチに統合すること
\item new\_branchでの作業をmasterに統合する場合,最初にmasterをcheckoutする
\end{itemize}
\end{block}

\begin{block}{コマンド操作}
\begin{minted}[]{bash}
git checkout master
git merge new_branch
\end{minted}
\end{block}
\end{frame}
\begin{frame}[label=sec-3-2-4]{Conflict(競合)とその解消}
\begin{block}{Conflictとは}
\begin{itemize}
\item branchで行う作業がかち合った場合,発生する
\item mergeする際,conflictが生じた場合,エラーになる
\end{itemize}
\end{block}

\begin{block}{解消方法}
\begin{itemize}
\item エディタ等で編集を行い,解消する
\end{itemize}
\end{block}

\begin{block}{参考文献}
\begin{itemize}
\item \href{https://help.github.com/articles/resolving-a-merge-conflict-from-the-command-line}{Resolving a merge conflict from the command line · GitHub Help}
\end{itemize}
\end{block}
\end{frame}

\section{リモートのブランチ}
\label{sec-3-3}
\begin{frame}[fragile,label=sec-3-3-1]{BranchのPush}
 \begin{block}{リモートへのPush}
\begin{itemize}
\item BranchをGitHubにPushすることができる
\item masterブランチをPushした際と同様,upstreamを指定する
\item PushできたかどうかをWebブラウザで確認する
\end{itemize}
\end{block}
\begin{block}{コマンド}
\begin{minted}[]{bash}
git push --set-upstream origin new_branch
\end{minted}
\end{block}
\end{frame}
\section{Pull Request}
\label{sec-3-4}
\begin{frame}[fragile,label=sec-3-4-1]{Pull Requestの作成}
 \begin{block}{Pull Roquestとは?}
\begin{itemize}
\item pushしたbranchでの作業の統合(merge)を依頼する
\item hubコマンドの \texttt{pull-request} で発行できる
\end{itemize}
\end{block}
\begin{block}{コマンド}
\begin{minted}[]{bash}
git pull-request -m 'Update a new branch'
\end{minted}
\end{block}
\end{frame}
\begin{frame}[fragile,label=sec-3-4-2]{Pull Requestのmerge}
 \begin{block}{Pull Requestをレビューする}
\begin{itemize}
\item WebブラウザでPull Requestを確認する
\end{itemize}
\end{block}

\begin{block}{ブラウザでmerge}
\begin{itemize}
\item 問題なければmergeボタンを押す
\end{itemize}
\end{block}

\begin{block}{コマンドラインでmergeする場合}
\begin{minted}[]{bash}
git merge pull_request_URL
\end{minted}
\end{block}
\end{frame}
\begin{frame}[fragile,label=sec-3-4-3]{BranchのPull}
 \begin{block}{BranchをPullするとは}
\begin{itemize}
\item リモートで行われた変更を適用すること
\item 内部的にはfetchでダウンロードしてからmergeする
\end{itemize}
\end{block}

\begin{block}{コマンド}
\begin{minted}[]{bash}
git checkout master
git pull
\end{minted}
\end{block}
\end{frame}
\section{<演習課題3>}
\label{sec-3-5}
\begin{frame}[fragile,label=sec-3-5-1]{branchの操作(ローカル)}
 \begin{block}{演習課題}
\begin{enumerate}
\item \texttt{my\_enpit} リポジトリでブランチを作成しなさい( \texttt{new\_branch} )
\item \texttt{checkout} で \texttt{new\_branch} に移動する
\item ファイルを編集しcommitする
\item \texttt{master} ブランチに移動してファイルの内容が
「編集されていないこと」を確認しなさい
\item \texttt{merge} して,変更を適用しなさい
\end{enumerate}
\end{block}
\end{frame}

\begin{frame}[fragile,label=sec-3-5-2]{競合の発生と解消}
 \begin{block}{演習課題}
\begin{enumerate}
\item \texttt{new\_branch} でファイルを編集して,commitする
\item \texttt{master} に移動し,ファイルの同じ箇所を編集して,commitする
\item \texttt{master} に \texttt{new\_branch} をmergeして,コンフリクトを発生させる
\item エディタで競合箇所を修正してcommitする
\end{enumerate}
\end{block}
\end{frame}

\begin{frame}[fragile,label=sec-3-5-3]{リモートのbranchの操作}
 \begin{block}{演習課題}
\begin{enumerate}
\item 新しいブランチを作成して,remoteにpushする
\item Pull Requestを送る
\item ブラウザで,Pull Requestをマージする
\item \texttt{master} ブランチに移動して, \texttt{pull} することで,更新する
\end{enumerate}
\end{block}
\end{frame}

\part{第4章 GitHubによる協同作業}
\label{sec-4}
\section{他の人の開発状況を見る}
\label{sec-4-1}
\begin{frame}[fragile,label=sec-4-1-1]{リモートのリポジトリをClone}
 \begin{block}{Cloneとは}
\begin{itemize}
\item GitHubで公開されているリポジトリはだれでも複製(clone)できる
\item ソースコードはローカルにコピーされ,閲覧やコンパイルなどができるようになる
\item アクセス権限がない場合は,pushできない
\end{itemize}
\end{block}

\begin{block}{コマンド}
\begin{minted}[]{bash}
git clone octocat/Spoon-Knife
\end{minted}
\end{block}
\end{frame}
\begin{frame}[fragile,label=sec-4-1-2]{Pull Requestをチェックアウト}
 \begin{block}{Pull Requestのチェックアウト}
\begin{itemize}
\item 誰かが作成したPull Requestの内容を,ブランチとしてローカルにコピーする
\item 試しに動作させたり,コードをチェックするときなどに利用
\end{itemize}
\end{block}

\begin{block}{コマンド}
\begin{minted}[]{bash}
git checkout https://github.com/octocat/Spoon-Knife/pull/3166
\end{minted}
\end{block}
\end{frame}
\section{開発に参加する}
\label{sec-4-2}
\begin{frame}[fragile,label=sec-4-2-1]{オリジナルのリポジトリをForkする}
 \begin{block}{Forkとは}
\begin{itemize}
\item Cloneしたリポジトリを,
自分のアカウントが所持するリポジトリとして
GitHub上で複製する
\item \texttt{remote} の値は,オリジナルのリポジトリが \texttt{origin} ,
自分のリポジトリは自分のGitHubユーザ名になる
\end{itemize}
\end{block}

\begin{block}{コマンド}
\begin{minted}[]{bash}
git fork
git remote -vv
\end{minted}
\end{block}
\end{frame}
\begin{frame}[fragile,label=sec-4-2-2]{ブランチを作成し自分のリポジトリにpush}
 \begin{block}{オリジナルの改変等}
\begin{itemize}
\item 新しい機能追加等を行う場合,ブランチを作成する
\item ブランチは,自分のリポジトリにpushする
\end{itemize}
\end{block}

\begin{block}{コマンド}
\begin{minted}[]{bash}
git branch my_branch
git checkout my_branch
<編集>
git commit -a -m 'Update'
git push -u ychubachi my_branch
\end{minted}
\end{block}
\end{frame}
\begin{frame}[fragile,label=sec-4-2-3]{Forkした元にPull Requestを送る}
 \begin{block}{コードのレビューやマージを依頼する}
\begin{itemize}
\item 新しい機能ができたら,オリジナルにPull Requestを送り,
レビューやマージをしてもらう
\end{itemize}
\end{block}

\begin{block}{コマンド}
\begin{minted}[]{bash}
git pull_request -m 'Pull Request'
\end{minted}
\end{block}
\end{frame}
\section{GitHubの他の機能}
\label{sec-4-3}
\begin{frame}[label=sec-4-3-1]{Issue/Wiki}
\begin{block}{Issue}
\begin{itemize}
\item 課題管理(ITS: Issue Tracking System)
\item コミットのメッセージでcloseできる
\begin{itemize}
\item \href{https://help.github.com/articles/closing-issues-via-commit-messages}{Closing issues via commit messages · GitHub Help}
\end{itemize}
\end{itemize}
\end{block}

\begin{block}{Wiki}
\begin{itemize}
\item GitHubのリポジトリにWikiを作る
\begin{itemize}
\item \href{https://help.github.com/articles/about-github-wikis}{About GitHub Wikis · GitHub Help}
\end{itemize}
\end{itemize}
\end{block}
\end{frame}

\begin{frame}[label=sec-4-3-2]{GitHub}
\begin{block}{GitHub Pages}
\begin{itemize}
\item 特殊なブランチを作成すると,Webページが構築できる
\begin{itemize}
\item \href{https://pages.github.com/}{GitHub Pages}
\end{itemize}
\end{itemize}
\end{block}

\begin{block}{Git brame}
\begin{itemize}
\item だれがどの作業をしたかわかる(誰がバグを仕込んだのかも)
\begin{itemize}
\item \href{https://help.github.com/articles/using-git-blame-to-trace-changes-in-a-file}{Using git blame to trace changes in a file · GitHub Help}
\end{itemize}
\end{itemize}
\end{block}
\end{frame}

\section{<演習課題4-1>}
\label{sec-4-4}
\begin{frame}[fragile,label=sec-4-4-1]{our\_enpitにファイルを追加する}
 \begin{block}{演習課題}
\begin{enumerate}
\item \texttt{ychubich/our\_enpit} をcloneしてforkする
\item 新しいブランチを作成し,新規にファイルを追加する
\begin{itemize}
\item 内容は任意(自己紹介など)
\item Markdownで書いてください(拡張子は.md)
\end{itemize}
\item コミットを作成し,pull requestを送信する
\item 教員がマージ作業を行います
\end{enumerate}
\end{block}
\end{frame}

\begin{frame}[label=sec-4-4-2]{既存のファイルを変更する}
\begin{block}{演習課題}
\begin{enumerate}
\item README.mdを改変して,pull requestを送信する
\item GitHubのPull Request一覧を確認する
\item おそらくコンフリクトが発生するので,
GitHubの指示に従い競合を解消する
\end{enumerate}
\end{block}
\end{frame}

\section{<演習課題4-2>}
\label{sec-4-5}
\begin{frame}[label=sec-4-5-1]{隣の人との協同作業}
\begin{block}{演習課題}
\begin{enumerate}
\item ここまでの演習ができた人と,できていない人とでペアを組む
\begin{itemize}
\item できていない人ができた人の隣の席に移動する
\end{itemize}
\item 新しくリポジトリを作成する(名称は任意)
\item 互いに,隣の席の人にリポジトリ名を教え,forkしてもらい
Pull Requestを送ってもらう
\item マージしてあげる
\item 2〜3を繰り返し,協同作業を行ってみよう
\end{enumerate}
\end{block}
\end{frame}

\section{<演習課題4-3>}
\label{sec-4-6}
\begin{frame}[label=sec-4-6-1]{Issue/Wikiの利用}
\begin{block}{演習課題}
\begin{itemize}
\item GitHubのIssueの機能を使ってみなさい
\item commitのログでIssueをクローズさせてみなさい
\item Wikiを作ってください
\end{itemize}
\end{block}
\end{frame}

\part{第5章 Sinatraアプリの開発}
\label{sec-5}
\section{Sinatraアプリケーションの作成}
\label{sec-5-1}
\begin{frame}[label=sec-5-1-1]{Sinatraを使った簡単なWebアプリケーション}
\begin{block}{Sinatraとは?}
\begin{itemize}
\item Webアプリケーションを作成するDSL
\item Railsに比べ軽量で,学習曲線が緩やか
\end{itemize}
\end{block}

\begin{block}{参考文献}
\begin{itemize}
\item \href{http://www.sinatrarb.com/}{Sinatra}
\end{itemize}
\end{block}
\end{frame}
\begin{frame}[fragile,label=sec-5-1-2]{Sinatraアプリ用リポジトリを作成する}
 \begin{block}{内容}
\begin{itemize}
\item Sinatraアプリを作成するため,新しいリポジトリを作る
\end{itemize}
\end{block}

\begin{block}{コマンド}
\begin{minted}[]{bash}
mkdir ~/sinatra_enpit
cd ~/shinatra_enpit
git init
git create
\end{minted}
\end{block}
\end{frame}
\begin{frame}[fragile,label=sec-5-1-3]{Sinatraアプリを作成する}
 \begin{block}{コマンド}
\begin{minted}[]{bash}
emacs hello.rb
git add hello.rb
git commit -m 'Create hello.rb'
\end{minted}
\end{block}
\begin{block}{コード: \texttt{hello.rb}}
\begin{minted}[]{ruby}
require 'sinatra'

get '/' do
  "Hello World!"
end
\end{minted}
\end{block}
\end{frame}
\begin{frame}[fragile,label=sec-5-1-4]{Sinatraアプリを起動する}
 \begin{block}{起動の方法}
\begin{itemize}
\item hello.rbをrubyで動かせば,サーバが立ち上がる
\item vagrantのport forwardを利用するためのオプションを追加する
\begin{itemize}
\item \href{http://stackoverflow.com/questions/21250885/unable-to-access-sinatra-app-on-host-machine-with-vagrant-forwarded-ports}{ruby - Unable to access Sinatra app on host machine with Vagrant forwarded ports - Stack Overflow}
\end{itemize}
\end{itemize}
\end{block}
\begin{block}{コマンド}
\begin{minted}[]{bash}
ruby hello.rb -o 0.0.0.0
\end{minted}
\end{block}
\end{frame}
\begin{frame}[label=sec-5-1-5]{Sinatraアプリの動作確認}
\begin{block}{動作確認の方法}
\begin{itemize}
\item Host OSのWebブラウザで,\url{http://localhost:4567} にアクセスする.
\end{itemize}
\end{block}
\end{frame}
\section{Herokuでアプリケーションを動かす}
\label{sec-5-2}
\begin{frame}[fragile,label=sec-5-2-1]{コマンドラインでHerokuにログインする}
 \begin{block}{内容}
\begin{itemize}
\item enPiT環境には \texttt{heroku} コマンドをインストールしてある
\item \texttt{heroku} コマンドを用いて,Herokuにログインできる
\item 以後の作業はHerokuコマンドを利用する
\end{itemize}
\end{block}

\begin{block}{コマンド}
\begin{minted}[]{bash}
heroku login
\end{minted}
\end{block}
\end{frame}
\begin{frame}[fragile,label=sec-5-2-2]{herokuにSSHの公開鍵を設定する}
 \begin{block}{内容}
\begin{itemize}
\item Herokuもgitのリモートリポジトリである
\item ここに公開鍵でアクセスできるようにする
\end{itemize}
\end{block}

\begin{block}{コマンド}
\begin{minted}[]{bash}
heroku keys:add
\end{minted}
\end{block}

\begin{block}{確認}
\begin{minted}[]{bash}
heroku keys
\end{minted}
\end{block}
\end{frame}
\begin{frame}[fragile,label=sec-5-2-3]{Herokuで動作できるSinatraアプリ}
 \begin{block}{内容}
\begin{itemize}
\item Herokuで動作できるSinatraアプリと設定ファイルの例
\begin{itemize}
\item \href{https://devcenter.heroku.com/articles/rack#sinatra}{Deploying Rack-based Apps | Heroku Dev Center}
\end{itemize}
\item 例を見ながら,エディタを用いて,次の3つのファイルを作成する
\begin{description}
\item[{\texttt{hello.rb}}] RubyによるWebアプリ本体(作成済み)
\item[{\texttt{config.ru}}] Webアプリサーバ(Rack)の設定
\item[{\texttt{Gemfile}}] アプリで利用するライブラリ(Gem)
\end{description}
\end{itemize}
\end{block}

\begin{block}{コマンド}
\begin{minted}[]{bash}
emacs config.ru
emacs Gemfile
\end{minted}
\end{block}
\end{frame}
\begin{frame}[fragile,label=sec-5-2-4]{Bundle install}
 \begin{block}{内容}
\begin{itemize}
\item \texttt{Gemfile} の中身に基づき,必要なGem(ライブラリ)をダウンロードする
\begin{itemize}
\item \texttt{Gemfile.lock} というファイルができる
\item このファイルもcommitの対象に含める
\end{itemize}
\end{itemize}
\end{block}

\begin{block}{コマンド}
\begin{minted}[]{bash}
bundle install
\end{minted}
\end{block}
\end{frame}
\begin{frame}[fragile,label=sec-5-2-5]{アプリをGitHubにpushする}
 \begin{block}{内容}
\begin{itemize}
\item Herokuで動かす前に,commitが必要
\item ついでに,GitHubにコードをpushしておく
\begin{itemize}
\item この場合のpush先は \texttt{origin master}
\end{itemize}
\end{itemize}
\end{block}

\begin{block}{コマンド}
\begin{minted}[]{bash}
git add .
git commit -m 'Add configuration files for Heroku'
git push -u origin master
\end{minted}
\end{block}
\end{frame}
\begin{frame}[fragile,label=sec-5-2-6]{Herokuにアプリを作る}
 \begin{block}{アプリを作る}
\begin{itemize}
\item Herokuが自動生成したURLが表示されるので,メモする
\item \texttt{git remote -v} でherokuという名前のremoteが追加されたことが分かる
\item WebブラウザでHerokuの管理画面を開くと,アプリができていることが確認できる
\end{itemize}
\end{block}
\begin{block}{コマンド}
\begin{minted}[]{bash}
heroku create
git remote -v
\end{minted}
\end{block}
\end{frame}
\begin{frame}[fragile,label=sec-5-2-7]{Herokuにアプリを配備する}
 \begin{block}{配備する方法}
\begin{itemize}
\item Herokuのリモートリポジトリにpushする
\item WebブラウザでアプリのURLを開き,動作を確認する
\end{itemize}
\end{block}

\begin{block}{コマンド}
\begin{minted}[]{bash}
git push heroku master
\end{minted}
\end{block}
\end{frame}
\section{<演習課題5-1>}
\label{sec-5-3}
\begin{frame}[label=sec-5-3-1]{Sinatraアプリの作成}
\begin{block}{演習課題}
\begin{itemize}
\item Sinatraアプリを作成して,Herokuで動作させなさい
\item SinatraのDSLについて調べ,機能を追加しなさい
\item コミットのログは詳細に記述し,どんな作業を行ったかが
他の人にも分かるようにしなさい
\item 完成したコードはGitHubにもpushしなさい
\end{itemize}
\end{block}
\end{frame}

\section{<演習課題5-2> (オプション)}
\label{sec-5-4}
\begin{frame}[label=sec-5-4-1]{Sinatraアプリの共同開発}
\begin{block}{演習課題}
\begin{itemize}
\item 隣の席の人と協同でSinatraアプリを開発しなさい
\item 一方がGitHubのリポジトリを作成し,もう一人がForkする
\item 最初に,どんな機能をもたせるかを相談しなさい
\begin{itemize}
\item メンバーのスキルに合わせて,できるだけ簡単なもの
\item データベースは使わない
\end{itemize}
\item ブランチを作成し,Pull Requestを送る
\end{itemize}
\end{block}
\end{frame}
\part{第6章 Ruby on Railsアプリの開発}
\label{sec-6}
\section{Ruby on Railsアプリの生成と実行}
\label{sec-6-1}
\begin{frame}[label=sec-6-1-1]{RoRを使ったWebアプリケーション}
\begin{block}{Ruby on Rails(RoR)とは?}
\begin{itemize}
\item Webアプリケーションを作成するためのフレームワーク
\end{itemize}
\end{block}

\begin{block}{参考文献}
\begin{itemize}
\item \href{http://rubyonrails.org/}{Ruby on Rails}
\end{itemize}
\end{block}
\end{frame}
\begin{frame}[label=sec-6-1-2]{Herokuで動かす方法}
\begin{block}{Getting Started}
\begin{itemize}
\item \href{https://devcenter.heroku.com/articles/getting-started-with-rails4}{Getting Started with Rails 4.x on Heroku | Heroku Dev Center}
\end{itemize}
\end{block}
\begin{block}{DBについて}
\begin{itemize}
\item DatabeseはPostgreSQLを使用する
\begin{itemize}
\item RoR標準のsqliteは使わない
\end{itemize}
\end{itemize}
\end{block}
\end{frame}
\begin{frame}[fragile,label=sec-6-1-3]{PostgreSQLにDBを作成}
 \begin{block}{開発で利用するDB}
\begin{description}
\item[{rails\_enpit\_development}] 開発作業中に利用
\item[{rails\_enpit\_test}] テスト用に利用
\item[{rails\_enpit\_production}] 本番環境で利用(ローカルには作成しない)
\end{description}
\end{block}

\begin{block}{コマンド}
\begin{minted}[]{bash}
createdb rails_enpit_development
createdb rails_enpit_test
\end{minted}
\end{block}
\end{frame}
\begin{frame}[fragile,label=sec-6-1-4]{\texttt{rails\_enpit} リポジトリを作成する}
 \begin{block}{内容}
\begin{itemize}
\item \texttt{rails} は予め,仮想化環境にインストールしてある
\item \texttt{rails new} コマンドを用いて,RoRアプリの雛形を作成する
\end{itemize}
\end{block}

\begin{block}{コマンド}
\begin{minted}[]{bash}
rails new ~/rails_enpit --database=postgresql
cd ~/rails_enpit
git init
git create
git add .
git commit -m 'Generate a new rails app'
git push -u origin master
\end{minted}
\end{block}
\end{frame}
\begin{frame}[fragile,label=sec-6-1-5]{Gemfileの変更}
 \begin{block}{変更する内容}
\begin{itemize}
\item GemfileにRails内部で動作するJavaScriptの実行環境を設定する
\item 当該箇所のコメントを外す
\item 変更をcommitしておく
\end{itemize}
\end{block}

\begin{block}{変更前}
\begin{minted}[]{ruby}
# gem 'therubyracer',  platforms: :ruby
\end{minted}
\end{block}
\begin{block}{変更後}
\begin{minted}[]{ruby}
gem 'therubyracer',  platforms: :ruby
\end{minted}
\end{block}
\end{frame}
\begin{frame}[fragile,label=sec-6-1-6]{Bundle installの実行}
 \begin{block}{\texttt{bundle install}}
\begin{itemize}
\item Gemfileを読み込み,必要なgemをインストールする
\item \texttt{rails new} をした際にも, \texttt{bundle install} は実行されている
\item 今回はtherubyracerと,それが依存しているgemでまだインストールしていないものをインストール
\item インストールする先は \texttt{\textasciitilde{}/.rbenv} 以下の特定のディレクトリ
\end{itemize}
\end{block}

\begin{block}{コマンド}
\begin{minted}[]{bash}
bundle install
git commit -a -m 'Run bundle install'
\end{minted}
\end{block}
\end{frame}
\begin{frame}[fragile,label=sec-6-1-7]{Rails serverの起動}
 \begin{block}{Rails serverを起動}
\begin{itemize}
\item この段階で,アプリケーションを起動できるようになっている
\item Host OSのWebブラウザで, \texttt{http://localhost:3000} にアクセスして確認
\item 端末にもログが表示される
\item 確認したら,端末でCtrl-Cを押してサーバを停止する
\end{itemize}
\end{block}

\begin{block}{コマンド}
\begin{minted}[]{bash}
rails server
\end{minted}
\end{block}
\end{frame}
\section{Controller/Viewの作成}
\label{sec-6-2}
\begin{frame}[fragile,label=sec-6-2-1]{Hello Worldを表示するController}
 \begin{block}{Controllerとは?}
\begin{itemize}
\item MVC構造でいうController
\item HTTPのリクエストを処理し,Viewに引き渡す
\item \texttt{rails generate controller} コマンドで作成する
\end{itemize}
\end{block}

\begin{block}{コマンド}
\begin{minted}[]{bash}
rails generate controller welcome
\end{minted}
\end{block}
\end{frame}

\begin{frame}[fragile,label=sec-6-2-2]{Viewの作成}
 \begin{block}{Viewとは?}
\begin{itemize}
\item HTML等で結果をレンダリングして表示する
\item \texttt{app/views/welcome/index.html.erb} を作成する
\item erbで作成するのが一般的で,内部でRubyコードを動作させることができる
\end{itemize}
\end{block}

\begin{block}{\texttt{index.html.erb}}
\begin{minted}[]{html}
<h2>Hello World</h2>
<p>
  The time is now: <%= Time.now %>
</p>
\end{minted}
\end{block}
\end{frame}
\begin{frame}[fragile,label=sec-6-2-3]{rootとなるrouteの設定}
 \begin{block}{Routeとは?}
\begin{itemize}
\item HTTPのリクエスト(URL)とコントローラを紐付ける設定
\item ここでは \texttt{root} へのリクエスト( \texttt{GET /} )を \texttt{welcome} コントローラの \texttt{index} メソッドに紐付ける
\item \texttt{rake routes} で確認する
\end{itemize}
\end{block}

\begin{block}{\texttt{config/routes.rb} の当該箇所をアンコメント}
\begin{minted}[]{ruby}
root 'welcome#index'
\end{minted}
\end{block}
\end{frame}
\begin{frame}[fragile,label=sec-6-2-4]{ControllerとViewの動作確認}
 \begin{block}{動作確認の方法}
\begin{itemize}
\item 再度, \texttt{rails server} でアプリを起動する
\item Webブラウザで \texttt{http://localhost:3000/} を開いて確認する
\end{itemize}
\end{block}

\begin{block}{コマンド}
\begin{minted}[]{bash}
rails server
\end{minted}
\end{block}
\end{frame}
\begin{frame}[fragile,label=sec-6-2-5]{ここまでをコミットしておく}
 \begin{block}{ここまでの内容}
\begin{itemize}
\item ここまでの作業で,controllerとviewを1つ備えるRoRアプリができた
\item 作業が一区切りしたので,commitする
(commitはひとかたまりの作業に対して行う)
\end{itemize}
\end{block}

\begin{block}{コマンド}
\begin{minted}[]{bash}
git add .
git commit -m 'Create welcome controller and view'
\end{minted}
\end{block}
\end{frame}
\section{Herokuにディプロイする}
\label{sec-6-3}
\begin{frame}[fragile,label=sec-6-3-1]{Gemfileの設定}
 \begin{block}{Heroku用Gem}
\begin{itemize}
\item \texttt{Gemfile} に \texttt{rails\_12factor} を追加する
\item Rubyのバージョンも指定しておく
\item \texttt{Gemfile} を変更したら必ず \texttt{bundle install} すること
\end{itemize}
\end{block}
\begin{block}{\texttt{Gemfile} に追加する内容}
\begin{minted}[]{ruby}
gem 'rails_12factor', group: :production
ruby '2.1.2'
\end{minted}
\end{block}
\end{frame}
\begin{frame}[fragile,label=sec-6-3-2]{Gitにコミット}
 \begin{block}{コミットする必要性}
\begin{itemize}
\item Herokuにコードを送るには,gitを用いる
\item ローカルで最新版をcommitしておく必要がある
\item ついでにGitHubにもpushしておく
\end{itemize}
\end{block}

\begin{block}{コマンド}
\begin{minted}[]{bash}
git commit -a -m 'Set up for Heroku'
git push # origin master -> GitHub が省略されている
\end{minted}
\end{block}
\end{frame}
\begin{frame}[fragile,label=sec-6-3-3]{Herokuアプリの作成とディプロイ}
 \begin{block}{作成とディプロイ}
\begin{itemize}
\item \texttt{heroku} コマンドを利用してアプリを作成する
\item \texttt{heroku create} で表示されたURLを開く
\item \texttt{git push} でディプロイすると,Herokuからのログが流れてくる
\end{itemize}
\end{block}

\begin{block}{コマンド}
\begin{minted}[]{bash}
heroku create
git push heroku master
\end{minted}
\end{block}
\end{frame}
\section{<演習課題6>}
\label{sec-6-4}
\begin{frame}[fragile,label=sec-6-4-1]{RoRアプリの作成}
 \begin{block}{演習課題}
\begin{itemize}
\item ここまでの説明に従い,Herokuで動作するRoRアプリ( \texttt{rails\_enpit} )を完成させなさい
\end{itemize}
\end{block}
\end{frame}
\part{第7章 DBを使うアプリの開発と継続的統合}
\label{sec-7}
\section{DBとScaffoldの作成}
\label{sec-7-1}
\begin{frame}[fragile,label=sec-7-1-1]{Scaffold}
 \begin{block}{Scaffoldとは}
\begin{itemize}
\item \href{https://www.google.co.jp/search?q=scaffold&client=ubuntu&hs=PiK&channel=fs&hl=ja&source=lnms&tbm=isch&sa=X&ei=smUdVKaZKY7s8AXew4LwDw&ved=0CAgQ_AUoAQ&biw=1195&bih=925}{scaffold - Google 検索}
\item RoRでは,MVCの雛形を作る
\begin{itemize}
\item CRUD処理が全て実装される
\end{itemize}
\item 多くのコードが自動生成されるので,branchを切っておくと良い
\begin{itemize}
\item 動作が確認できたらbranchをマージ
\item うまく行かなかったらbranchごと削除すれば良い
\end{itemize}
\end{itemize}
\end{block}

\begin{block}{コマンド}
\begin{minted}[]{bash}
git branch books
git checkout books
rails generate scaffold book title:string author:string
\end{minted}
\end{block}
\end{frame}
\begin{frame}[fragile,label=sec-7-1-2]{DBのMigrate}
 \begin{block}{migrateとは}
\begin{itemize}
\item Databaseのスキーマ定義の更新
\item Scaffoldを追加したり,属性を追加したりした際に行う
\end{itemize}
\end{block}

\begin{block}{コマンド}
\begin{minted}[]{bash}
rake db:migrate
\end{minted}
\end{block}
\end{frame}
\begin{frame}[fragile,label=sec-7-1-3]{routeの確認}
 \begin{block}{route}
\begin{itemize}
\item ルーティングの設定を確認しよう
\end{itemize}
\end{block}

\begin{block}{コマンド}
\begin{minted}[]{bash}
rake routes
\end{minted}
\end{block}
\end{frame}
\begin{frame}[fragile,label=sec-7-1-4]{動作確認}
 \begin{block}{動作確認の方法}
\begin{itemize}
\item Webブラウザで \url{http://localhost:3000/books} を開く
\item CRUD処理が完成していることを確かめる
\end{itemize}
\end{block}

\begin{block}{コマンド}
\begin{minted}[]{bash}
rails server
\end{minted}
\end{block}
\end{frame}
\begin{frame}[fragile,label=sec-7-1-5]{完成したコードをマージ}
 \begin{block}{ブランチをマージ}
\begin{itemize}
\item 動作確認できたので, \texttt{books} branchをマージする
\item 不要になったブランチは, \texttt{git branch -d} で削除する
\end{itemize}
\end{block}

\begin{block}{コマンド}
\begin{minted}[]{bash}
git add .
git commit -m 'Generate books scaffold'
git checkout master
git merge books
git branch -d books
\end{minted}
\end{block}
\end{frame}
\begin{frame}[fragile,label=sec-7-1-6]{Herokuにディプロイ}
 \begin{block}{ディプロイ}
\begin{itemize}
\item ここまでのアプリをディプロイする
\item herokuにあるdbもmigrateする
\item Webブラウザで動作確認する
\end{itemize}
\end{block}

\begin{block}{コマンド}
\begin{minted}[]{bash}
git push heroku master
heroku run rake db:migrate
\end{minted}
\end{block}
\end{frame}
\begin{frame}[fragile,label=sec-7-1-7]{Scaffoldの作成を取り消す場合(参考)}
 \begin{block}{取り消す操作}
\begin{itemize}
\item migrationを取り消す
\item branchに一旦コミットして,masterブランチに移動
\item branchを削除
\end{itemize}
\end{block}

\begin{block}{コマンド}
\begin{minted}[]{bash}
rake db:rollback
git add .
git commit -m 'Rollback'
git checkout master
git branch -D books
\end{minted}
\end{block}
\end{frame}
\begin{frame}[label=sec-7-1-8]{PostgereSQLクライアントのコマンド(参考)}
\begin{itemize}
\item psqlでDBにログイン
\end{itemize}

\begin{center}
\begin{tabular}{ll}
Backslashコマンド & 説明\\
\hline
l & DBの一覧\\
c & DBに接続\\
d & リレーションの一覧\\
q & 終了\\
\end{tabular}
\end{center}
\end{frame}

\section{RoRアプリのテスト}
\label{sec-7-2}
\begin{frame}[label=sec-7-2-1]{テストについて}
\begin{block}{ガイド}
\begin{itemize}
\item \href{http://guides.rubyonrails.org/testing.html}{A Guide to Testing Rails Applications — Ruby on Rails Guides}
\end{itemize}
\end{block}
\end{frame}
\begin{frame}[fragile,label=sec-7-2-2]{テストの実行}
 \begin{block}{テストコード}
\begin{itemize}
\item Scaffoldはテストコードも作成してくれる
\item テスト用のDB( \texttt{rails\_enpit\_test} )が更新される
\end{itemize}
\end{block}

\begin{block}{コマンド}
\begin{minted}[]{bash}
rake test
\end{minted}
\end{block}
\end{frame}
\section{Travis CIとの連携}
\label{sec-7-3}
\begin{frame}[label=sec-7-3-1]{Travis CIのアカウント作成}
\begin{block}{アカウントの作り方}
\begin{itemize}
\item 次のページにアクセスし,画面右上の「Sign in with GitHub」のボタンを押す
\begin{itemize}
\item \href{https://travis-ci.org/}{Travis CI - Free Hosted Continuous Integration Platform for the Open Source Community}
\end{itemize}
\item GitHubの認証ページが出るので,画面下部にある緑のボタンを押す
\item Travis CIから確認のメールが来るので,確認する
\end{itemize}
\end{block}

\begin{block}{Ruby アプリ}
− \href{http://docs.travis-ci.com/user/languages/ruby/}{Travis CI: Building a Ruby Project}
\end{block}
\end{frame}
\begin{frame}[fragile,label=sec-7-3-2]{Travisの初期化}
 \begin{block}{内容}
\begin{itemize}
\item Travisにログインして初期化を行う
\end{itemize}
\begin{itemize}
\item \texttt{init} すると \texttt{.travis.yml} ができる
\end{itemize}
\end{block}

\begin{block}{コマンド}

\begin{minted}[]{bash}
travis login --auto    # GitHubのログイン情報でログイン
travis init            # 質問には全てEnterを押す
\end{minted}
\end{block}
\end{frame}
\begin{frame}[fragile,label=sec-7-3-3]{Herokuとの連携}
 \begin{block}{Herokuとの連携}
\begin{itemize}
\item Travis CIからHerokuへの接続を設定する
\item master以外のブランチで実行すると,そのブランチのみHerokuに送る(ようだ)
\begin{itemize}
\item \href{http://docs.travis-ci.com/user/deployment/heroku/}{Travis CI: Heroku Deployment}
\end{itemize}
\end{itemize}
\end{block}

\begin{block}{コマンド}
\begin{minted}[]{bash}
travis setup heroku
\end{minted}
\end{block}
\end{frame}
\begin{frame}[fragile,label=sec-7-3-4]{Travisで動かすRubyのバージョン設定}
 \begin{block}{設定ファイルの変更}
\begin{itemize}
\item Rubyのバージョン
\end{itemize}
\end{block}

\begin{block}{.travis.yml(抜粋)}
\begin{minted}[]{yaml}
language: ruby
rvm:
- 2.1.2
\end{minted}
\end{block}
\end{frame}
\begin{frame}[fragile,label=sec-7-3-5]{Travis用DB設定ファイル}
 \begin{block}{TravisでのテストDB}
− テストDB用の設定ファイルを追加する
\end{block}

\begin{block}{\texttt{config/database.yml.travis}}
\begin{minted}[]{yaml}
test:
  adapter: postgresql
  database: travis_ci_test
  username: postgres
\end{minted}
\end{block}
\end{frame}
\begin{frame}[fragile,label=sec-7-3-6]{Travis上のDB設定}
 \begin{block}{設定ファイルの変更(追加)}
\begin{itemize}
\item PostgreSQLのバージョン
\item DBの作成
\item \href{http://docs.travis-ci.com/user/using-postgresql/}{Travis CI: Using PostgreSQl on Travis CI}
\end{itemize}
\end{block}
\begin{block}{.travis.yml(抜粋)}
\begin{minted}[]{yaml}
addons:
  postgresql: "9.3"
before_script:
  - psql -c 'create database travis_ci_test;' -U postgres
  - cp config/database.yml.travis config/database.yml
\end{minted}
\end{block}
\end{frame}
\begin{frame}[fragile,label=sec-7-3-7]{GitHubとTravis CI連携}
 \begin{block}{説明}
\begin{itemize}
\item ここまでの設定で,GitHubにpushされたコードは,
Travis CIでテストされ,テストが通ったコミットが
Herokuに送られるようになった
\item WebブラウザでTravis CIを開いて確認する
\end{itemize}
\end{block}

\begin{block}{コマンド}
\begin{minted}[]{bash}
git add .
git commit -m 'Configure Travis CI'
git push
\end{minted}
\end{block}
\end{frame}
\begin{frame}[label=sec-7-3-8]{Travis経由でのHerokuへのdeploy}
\begin{block}{Travisのログを閲覧}
\begin{itemize}
\item WebブラウザでTravis CIの画面を開く
\item ログを読む
\end{itemize}
\end{block}

\begin{block}{HerokuへのDeploy}
\begin{itemize}
\item テストが通れば,自動でHerokuに配備される
\item 配備できたらWebブラウザでアプリのページを開いて確認する
\end{itemize}
\end{block}
\end{frame}

\begin{frame}[fragile,label=sec-7-3-9]{補足:Sinatraでテストが通るようにする}
 \begin{block}{Gemfileに \texttt{rake} を追加する}
\begin{minted}[]{bash}
gem 'rake'
\end{minted}
\end{block}
\begin{block}{Rakefileを作成する}
\begin{minted}[]{ruby}
task :default => :test

require 'rake/testtask'

Rake::TestTask.new do |t|
  t.pattern = "./*_test.rb"
end
\end{minted}
\end{block}
\end{frame}
\section{<演習課題7>}
\label{sec-7-4}
\begin{frame}[fragile,label=sec-7-4-1]{Herokuへのdeploy}
 \begin{block}{演習課題}
\begin{itemize}
\item \texttt{rails\_enpit} にbooksスキャっフォルドを追加してHerokuに配備しなさい
\end{itemize}
\end{block}
\end{frame}

\begin{frame}[label=sec-7-4-2]{リンクの追加}
\begin{block}{演習課題}
\begin{itemize}
\item welcomeコントローラのviewから,
booksコントローラのviewへのリンクを追加しなさい
\end{itemize}
\end{block}
\end{frame}

\begin{frame}[label=sec-7-4-3]{Scaffoldの追加}
\begin{block}{演習課題}
\begin{itemize}
\item Scaffoldを追加しなさい
\item DBのmigrationを行い,動作確認しなさい
\item うまく動作したらHerokuに配備しなさい
\end{itemize}
\end{block}
\end{frame}

\begin{frame}[label=sec-7-4-4]{Travis経由でのHerokuへのdeploy}
\begin{block}{演習課題}
\begin{itemize}
\item Travis経由でHerokuへdeployできるようにする
\end{itemize}
\end{block}
\end{frame}
<<<<<<< HEAD
\begin{frame}[label=sec-7-4-5]{Status Image}
\begin{block}{演習課題}
\begin{itemize}
\item README.mdを編集し,Travisのテスト状況を表示するStatus Imageを追加する
\item \href{http://docs.travis-ci.com/user/status-images/}{Travis CI: Status Images}
\end{itemize}
\end{block}
\end{frame}
=======

>>>>>>> d35d84d85541f89a6cd5bf5abc516343597916a0
\part{第8章 Webアプリケーションの共同開発}
\label{sec-8}
\section{<演習課題8>}
\label{sec-8-1}
\begin{frame}[label=sec-8-1-1]{Webアプリケーションの共同開発}
\begin{itemize}
\item 4人(または3人)でチームを組み,1つのWebアプリケーションを開発しなさい
\begin{itemize}
\item 何を作るかは,チームで相談してください
\end{itemize}
\item 授業で取り上げたツールを使い,自由に試しなさい
\item 利用するフレームワークは,SinatraでもRailsでもどちらでもかまいません
\begin{itemize}
\item どちらを使うかは,チームで相談して決めてください
\end{itemize}
\end{itemize}
\end{frame}

\part{第9章 楽天APIを利用したアプリケーション}
\label{sec-9}
\section{楽天API}
\label{sec-9-1}
\begin{frame}[label=sec-9-1-1]{楽天APIとは?}
\begin{itemize}
\item \href{http://webservice.rakuten.co.jp/document/}{楽天ウェブサービス: API一覧}
\end{itemize}
\end{frame}
\begin{frame}[fragile,label=sec-9-1-2]{サンプルアプリ}
 \begin{itemize}
\item \href{https://github.com/ryuichiueda/rakuten_enpit_example}{Rakuten enPiT Example}
\begin{itemize}
\item \texttt{git clone} する
\end{itemize}
\item \texttt{bundle install} する
\item Herokuでアプリを作りアプリURLを取得
\begin{itemize}
\item \texttt{heroku create} する
\end{itemize}
\end{itemize}
\end{frame}
\begin{frame}[label=sec-9-1-3]{アプリIDの発行}
\begin{itemize}
\item 新規アプリを登録する
\begin{itemize}
\item \href{https://webservice.rakuten.co.jp/app/create}{楽天ウェブサービス: 新規アプリ登録}
\end{itemize}
\item アプリ名(任意),アプリのURL,認証コードを入力
\begin{itemize}
\item アプリID,アフィリエイトID等を控えておく
\end{itemize}
\end{itemize}
\end{frame}
\begin{frame}[fragile,label=sec-9-1-4]{環境変数の設定}
 \begin{itemize}
\item アプリID(APPID)とアフィリエイトID(AFID)を環境変数に登録
\item \texttt{\textasciitilde{}/.bash\_profile} に次の行を追加(自分のID等に書き換えること)
\item \texttt{exit} して,再度 \texttt{vagrant ssh}
\end{itemize}

\begin{minted}[]{bash}
export APPID=102266705971259xxxx
export AFID=11b23d92.8f6b6ff4.11b23d93.???????
\end{minted}
\end{frame}
\begin{frame}[fragile,label=sec-9-1-5]{ローカルでの動作確認}
 \begin{itemize}
\item ローカルで動作確認する
\end{itemize}

\begin{minted}[]{bash}
ruby hello.rb -o 0.0.0.0
\end{minted}
\end{frame}
\section{Herokuで動作させる}
\label{sec-9-2}
\begin{frame}[fragile,label=sec-9-2-1]{Herokuの環境変数}
 \begin{block}{環境変数の作成}
\begin{itemize}
\item 次のコマンドで,Heroku内部にも環境変数を作る
\item 参考
\begin{itemize}
\item \href{https://devcenter.heroku.com/articles/config-vars}{Configuration and Config Vars | Heroku Dev Center}
\end{itemize}
\end{itemize}
\end{block}

\begin{block}{コマンド}
\begin{minted}[]{bash}
heroku config:set APPID=102266705971259xxxx
heroku config:set AFID=11b23d92.8f6b6ff4.11b23d93.???????
\end{minted}
\end{block}
\end{frame}
\begin{frame}[fragile,label=sec-9-2-2]{Herokuでの動作確認}
 \begin{block}{内容}
\begin{itemize}
\item Herokuに直接Pushしてみる
\item webブラウザで動作確認
\end{itemize}
\end{block}
\begin{block}{コマンド}
\begin{minted}[]{bash}
git push heroku master
\end{minted}
\end{block}
\end{frame}
\section{Travis CI連携}
\label{sec-9-3}
\begin{frame}[fragile,label=sec-9-3-1]{.travis.ymlの再生成}
 \begin{block}{内容}
\begin{itemize}
\item \texttt{fork} して作業用のブランチを作成する
\item .travis.yml の削除と新規作成
\item 不要なRubyのバージョンを削除
\end{itemize}
\end{block}

\begin{block}{コマンド}
\begin{minted}[]{bash}
git fork
git branch new_feature
rm .travis.yml
travis init
travis heroku setup
emacs .travis.yml
\end{minted}
\end{block}
\end{frame}
\begin{frame}[fragile,label=sec-9-3-2]{Travis CIの環境変数}
 \begin{block}{内容}
\begin{itemize}
\item リポジトリで次のコマンドを打つ
\item 自分のAPPID,AFIDに書き換えること
\end{itemize}
\end{block}

\begin{block}{コマンド}
\begin{minted}[]{bash}
travis env set APPID 102266705971259xxxx
travis env set AFID 11b23d92.8f6b6ff4.11b23d93.???????
\end{minted}
\end{block}
\end{frame}
\begin{frame}[fragile,label=sec-9-3-3]{コミットしてpush}
 \begin{block}{内容}
\begin{itemize}
\item \texttt{add} して \texttt{commit}
\item 自分のリポジトリにpush
\end{itemize}
\end{block}

\begin{block}{コマンド}
\begin{minted}[]{bash}
git add .
git commit -m 'Update .travis.yml'
git push -u ychubach master
\end{minted}
\end{block}
\end{frame}
\section{<演習課題9-1>}
\label{sec-9-4}
\begin{frame}[label=sec-9-4-1]{ローカルでサンプルを動かす}
\begin{itemize}
\item 自分のAPPIDを作成する
\item 仮想化環境とHerokuの環境変数を設定
\item ローカルで動かしてみよう
\item Herokuに直接Pushして動かしてみよう
\end{itemize}
\end{frame}

\section{<演習課題9-2>}
\label{sec-9-5}
\begin{frame}[fragile,label=sec-9-5-1]{Travis経由で動かしてみよう(難易度:高)}
 \begin{itemize}
\item サンプルをTravis経由で動作させてみよう
\begin{itemize}
\item Forkして,自分のリポジトリにpushできるようにする
\item \texttt{.travis.yml} の設定を変更する
\begin{itemize}
\item やり方は各自で考えてみよう
\end{itemize}
\item Travis CIに環境変数を設定する
\end{itemize}
\end{itemize}
\end{frame}
\part{第10章 ミニプロジェクト}
\label{sec-10}
\section{<演習課題10>}
\label{sec-10-1}
\begin{frame}[label=sec-10-1-1]{ミニプロジェクト}
\begin{itemize}
\item 楽天APIを利用したWebアプリケーションを開発する
\begin{itemize}
\item 2つペアを統合し,4人のグループを編成する
\end{itemize}
\item グループで次のことを相談
\begin{itemize}
\item どんなアプリをつくるか
\item 役割分担
\end{itemize}
\item 授業で取り扱った内容のほか自分の知っている知識を活用してください
\begin{itemize}
\item JavaScript,CSS \ldots{}
\end{itemize}
\item README.md に使い方,HerokuのURLなどを書く
\item LICENCEも設定する
\end{itemize}
\end{frame}
\part{補足資料}
\label{sec-11}
\begin{frame}[label=sec-11-0-1]{.gitignoreについて}
\begin{itemize}
\item Gitに登録したくないファイルは.gitignoreに登録する
\item 例
\begin{itemize}
\item \href{https://github.com/github/gitignore/blob/master/Global/Emacs.gitignore}{gitignore/Emacs.gitignore at master · github/gitignore}
\end{itemize}
\end{itemize}
\end{frame}

\begin{frame}[fragile,label=sec-11-0-2]{HerokuのアプリのURL確認}
 \begin{minted}[]{bash}
heroku apps:info
\end{minted}
\end{frame}
\begin{frame}[fragile,label=sec-11-0-3]{Herokuのログをリアルタイムで見る}
 \begin{minted}[]{bash}
heroku logs --tail
\end{minted}
\end{frame}
\begin{frame}[fragile,label=sec-11-0-4]{\texttt{rails generate} などが動かない}
 \begin{itemize}
\item \href{https://devcenter.heroku.com/articles/getting-started-with-rails4#write-your-app}{Write your App}
\end{itemize}

\begin{minted}[]{bash}
spring stop
\end{minted}
\end{frame}
<<<<<<< HEAD
\begin{frame}[fragile,label=sec-9-0-6]{仮想環境内にファイル(画像など)}
=======
\begin{frame}[fragile,label=sec-11-0-5]{仮想環境内にファイル(画像など)}
>>>>>>> d35d84d85541f89a6cd5bf5abc516343597916a0
 \begin{itemize}
\item Guest OS内に \texttt{/vagrant} という共有フォルダがある
\item このフォルダはHost OSからアクセスできる
\item 場所はVagrantfileがあるフォルダ
\end{itemize}
\end{frame}
<<<<<<< HEAD
\part{楽天APIを用いた開発}
\label{sec-10}
\begin{frame}[label=sec-10-0-1]{演習課題}
\begin{itemize}
\item 楽天APIを利用したWebアプリケーションを開発する
\begin{itemize}
\item グループ編成
\end{itemize}
\item グループで次のことを相談
\begin{itemize}
\item どんなアプリをつくるか
\item 役割分担
\end{itemize}
\item 授業で取り扱った内容のほか自分の知っている知識を活用してください
\begin{itemize}
\item JavaScript,CSS \ldots{}
\end{itemize}
\end{itemize}
\end{frame}
\begin{frame}[label=sec-10-0-2]{演習課題}
\begin{itemize}
\item README.md に使い方,HerokuのURLなどを書く
\item LICENCEも設定する
\end{itemize}
\end{frame}
% Emacs 24.3.1 (Org mode 8.2.6)
\end{document}
=======
\begin{frame}[label=sec-11-0-6]{Status Image}
\begin{block}{演習課題}
\begin{itemize}
\item README.mdを編集し,Travisのテスト状況を表示するStatus Imageを追加する
\item \href{http://docs.travis-ci.com/user/status-images/}{Travis CI: Status Images}
\end{itemize}
\end{block}
\end{frame}
% Emacs 24.3.2 (Org mode 8.2.5h)
\end{document}
>>>>>>> d35d84d85541f89a6cd5bf5abc516343597916a0
